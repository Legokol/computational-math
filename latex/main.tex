\documentclass[a4paper,12pt]{article}

\usepackage{cmap}
\usepackage[T2A]{fontenc}
\usepackage[utf8]{inputenc}
\usepackage[english,russian]{babel}

\usepackage{amsmath, amssymb, amsthm}

\usepackage{indentfirst} %Красная строка
\usepackage[a4paper,top=1.3cm,bottom=2cm,left=1.5cm,right=1.5cm,marginparwidth=0.75cm]{geometry}
\usepackage{hyperref}
\hypersetup{hidelinks}

\usepackage{graphicx}
\graphicspath{{images/}}

\renewcommand{\phi}{\varphi}
\renewcommand{\epsilon}{\varepsilon}

\title{Практические задачи по вычислительной математике. Первое задание.}
\author{Николай Чусовитин, группа Б03-905}
\date{}

\begin{document}

\maketitle

\section*{Задача I.8.19}

Для вычисления функции $u = f(t)$ используется частичная сумма ряда Маклорена:

\begin{equation*}
    u(t) \approx u(0) + \frac{u'(0)}{1!} t + \dots + \frac{u^{(n)} (0)}{n!} t^n
\end{equation*}

\noindent
где аргумент задан с погрешностью $\Delta t = 10^{-3}$. При каком $n$ погрешность $u(t)$ не превышает $\Delta t$? Рассмотреть отрезки $[0, 1]$ и $[10, 11]$. Найти более совершенный алгоритм для вычисления функций $u(t) =\sin t$ и $u(t) = e^t$ на втором отрезке.

\subsection*{Решение}

Ошибку метода оценим, воспользовавшись остаточным членом в форме Лагранжа:

\begin{equation*}
    | \Delta_\text{метода} | = \left| \frac{u^{(n + 1)} (\xi)}{(n + 1)!} t^{n + 1} \right|
\end{equation*}

Перейдём теперь к рассмотрению заданных функций. Для $\sin t$ остаточный член в форме Лагранжа имеет вид

\begin{equation*}
    \Delta_\text{метода} \leq \frac{t^{2n + 1}}{(2n + 1)!}
\end{equation*}

На отрезке $[0, 1]$ имеем $\Delta_\text{метода} \leq \dfrac{1}{(2n + 1)!}$ и для необходимой точности достаточно взять $n \geq 3$. На $[10, 11]$ $\Delta_\text{метода} \leq \dfrac{11^{2n + 1}}{(2n + 1)!}$ необходимо уже $n \geq 17$. Для уменьшения $n$ на втором отрезке можно сделать замену переменных: $t = 3 \pi + \tilde{t}$. Тогда $\sin t = - \sin \tilde{t}$, где $\tilde{t}$ лежит на подмножестве отрезка $[0.5, 1.6]$. (Для синуса $n$ -- число элементов в разложении, встречаются только нечетные).

Для экспоненты остаточный член имеет вид:

\begin{equation}
    \Delta_\text{метода} \leq \frac{e^t}{(n + 1)!} t^{n + 1}
\end{equation}

На отрезке $[0, 1]$ имеем $\Delta_\text{метода} \leq \dfrac{e}{(n + 1)!} t^{n + 1}$ и для необходимой точности достаточно взять $n \geq 6$. На $[0, 1]$  $\Delta_\text{метода} \leq \dfrac{e^{11} 11^{n + 1}}{(n + 1)!} t^{n + 1}$ и необходимо $n \geq 42$. Для уменьшения $n$ не втором отрезке можно сделать замену переменных $t = 10 + \tilde{t}$, тогда $e^t = e^{10} e^{\tilde{t}}$. Из-за лишнего множителя нужно будет увеличить число членов, но не так значительно ($n \geq 11$).

\section*{Задача IV.12.8 (б)}

Найти методом простой итерации полуширину на полувысоте с точностью $10^{-3}$ функции

\begin{equation*}
    f(x) = x e^{-x^2}, \; x \geq 0
\end{equation*}

\subsection*{Решение}

Найдём сначала максимум функции, приравняв производную к нулю:

\begin{equation*}
    e^{-x^2} - 2 x^2 e^{-x^2} = 0
\end{equation*}

Имеем: $x_m = 1 / \sqrt{2}, f_m = 1 / \sqrt{2 e}$. Чтобы найти полуширину нужно решить уравнение

\begin{equation*}
    f(x) - \frac{f_m}{2} = 0
\end{equation*}

\noindent
причём один его корень будет расположен на интервале $(0, x_m)$, а второй -- на $(x_m, 2)$. Воспользуемся итерационным методом:

\begin{align*}
    \phi_1 (x) &= \frac{f_m}{2} e^{x^2} \\
    x^{n + 1} &= \phi_1 (x^n)
\end{align*}

Для сходимости процесса необходимо, чтобы производная $\phi_1' (x)$ по модулю не превосходила единицы. Это условие выполняется на первом интервале ($\phi_1' < 1 / 2$), значит метод можно использовать для поиска левого корня. Для правого корня:

\begin{align*}
    \phi_2 (x) &= \sqrt{\ln \frac{2 x}{f_m}} \\
    x^{n + 1} &= \phi_2 (x^n)
\end{align*}

Производная $\phi_2' (x)$ не превосходит по модулю $1 / \sqrt{\ln (4 e)}$ на втором интервале.

Для определения полуширины нужно каждый из из корней найти с точностью $\epsilon / 2$. Для остановки воспользуемся критерием

\begin{equation*}
    \frac{|x^{n + 1} - x^n|}{1 - q} \leq \epsilon
\end{equation*}

\noindent
здесь $q$ -- число, ограничивающее производную. Результаты решения:

\begin{align*}
    x_1 &= 0.226 \\
    x_2 &= 1.359 \\
    \Delta_{1 / 2} &= x_2 - x_1 = 1.133
\end{align*}

\section*{VI.9.32}

По заданным значениям населения США в 1910-2000 годах построить а) интерполяционный полином в форме Ньютона б) кубический сплайн и, экстраполируя на 2010 год, сравнить полученные значения с точным 308 745 538 человек.

\begin{table*}[htbp]
    \centering
    \begin{tabular}{|c|c|c|}
        \hline
        Тип интерполяции & В форме Ньютона & Кубический сплайн \\
        \hline
        Население в 2010 году & 827 906 509 & 314 133 939 \\
        \hline
        Абсолютная ошибка & $5.19161 \cdot 10^8$ & $5.3884 \cdot 10^6$ \\
        \hline
        Относительная ошибка & 1.68152 & 0.0174526 \\
        \hline
    \end{tabular}
\end{table*}

Относительная ошибка интерполяционного многочлена в форме Ньютона составляет более 100\% от точного значения, что связано с высокой степенью многочлена (10 точек -- полином 9 степени), приводящей к быстрому росту за пределами отрезка интерполяции. Относительная ошибка кубического сплайна составляет около 2\%. Она значительно меньше ошибки полинома Ньютона ввиду меньшей степени многочлена.

\section*{Задача 1}

Найти все корни системы уравнений с точностью $10^{-6}$

\begin{equation*}
    \left\{
        \begin{aligned}
            x^2 + y^2 = 1 \\
            y = \tg x
        \end{aligned}
    \right.
\end{equation*}

\subsection*{Решение}

Первое уравнение системы -- уравнение окружности. Кривая $y = \tg x$ пересекает её в двух симметричных точках, так что решение можно искать на интервалах $(-1, 0)$ и $(0, 1)$. С учётом симметрии задачи достаточно найти одно из этих решений. Решать систему будем методом простой итерации:

\begin{equation*}
    \left\{
        \begin{aligned}
            x^{n + 1} &= \arctg y^n \\
            y^{n + 1} &= \sqrt{1 - (x^n)^2}
        \end{aligned}
    \right.
\end{equation*}

При начальном приближении $(x_0, y_0) = (0.5, 0.5)$ имеем решение системы:

\begin{align*}
    (x_1, y_1) &= (0.649889, \; 0.760029) \\
    (x_2, y_2) &= (-0.649889, \; -0.760029) \\
\end{align*}

\section*{Задача 2}

Вычислить интеграл

\begin{equation*}
    I = \int_0^3 \sin (100 x) e^{- x ^2} \cos (2 x) dx
\end{equation*}

\end{document}